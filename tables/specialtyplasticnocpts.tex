% latex table generated in R 4.0.1 by xtable 1.8-4 package
% Tue Jan 04 19:21:54 2022
\begin{table}[ht]
\centering
\begin{tabular}{rrrrrrrrrrrrrrrrrrrrrrrrrrrrrrrrrrrrrrrrrrrrrrrrrrrrrrrrrrrrrrrrrrrrrrrrrrrrrrrrrrrrrrrrrrrrrrrrrrrrrrrrrrrrrrrrrrrrrrrrrrrrrrrrrrrrrrrrrrrrrrrrrrrrrrrrrrrrrrrrrrrrrrrrrrrrrrrrrrrrrrrrrrrrrrrrrrrrrrrrrrrrrrrrrrrrrrrrrrrrrrrrrrrrrrrrrrrrrrrrrrrrrrrrrrrrrrrrrrrrrrrrrrrrrrrrrrrrrrrrrrrrrrrrrrrrrrrrrrrrrrrrrrrrrrrrrrrrrrrrrrrrrrrrrrrrrrrrrrrrrrrrrrrrrrrrrrrrrrrrrrrrrrrrrrrrrrrrrrrrrrrrrrrrrrrrrrrrrrrrrrrrrrrrrrrrrrrrrrrrrrrrrrrrrrrrrrrrrrrrrrrrrrrrrrrrrrrrrrrrrrrrrrrrrrrrrrrrrrrrrrrrrrrrrrrrrrrrrrrrrrrrrrrrrrrrrrrrrrrrrrrrrrrrrrrrrrrrrrrrrrrrrrrrrrrrrrrrrrrrrrrrrrrrrrrrrrrrrrrrrrrrrrrrrrrrrrrrrrrrrrrrrrrrrrrrrrrrrrrrrrrrrrr}
  \hline
 & 1 & 2 & 3 & 4 & 5 & 6 & 7 & 8 & 9 & 10 & 11 & 12 & 13 & 14 & 15 & 16 & 17 & 18 & 19 & 20 & 21 & 22 & 23 & 24 & 25 & 26 & 27 & 28 & 29 & 30 & 31 & 32 & 33 & 34 & 35 & 36 & 37 & 38 & 39 & 40 & 41 & 42 & 43 & 44 & 45 & 46 & 47 & 48 & 49 & 50 & 51 & 52 & 53 & 54 & 55 & 56 & 57 & 58 & 59 & 60 & 61 & 62 & 63 & 64 & 65 & 66 & 67 & 68 & 69 & 70 & 71 & 72 & 73 & 74 & 75 & 76 & 77 & 78 & 79 & 80 & 81 & 82 & 83 & 84 & 85 & 86 & 87 & 88 & 89 & 90 & 91 & 92 & 93 & 94 & 95 & 96 & 97 & 98 & 99 & 100 & 101 & 102 & 103 & 104 & 105 & 106 & 107 & 108 & 109 & 110 & 111 & 112 & 113 & 114 & 115 & 116 & 117 & 118 & 119 & 120 & 121 & 122 & 123 & 124 & 125 & 126 & 127 & 128 & 129 & 130 & 131 & 132 & 133 & 134 & 135 & 136 & 137 & 138 & 139 & 140 & 141 & 142 & 143 & 144 & 145 & 146 & 147 & 148 & 149 & 150 & 151 & 152 & 153 & 154 & 155 & 156 & 157 & 158 & 159 & 160 & 161 & 162 & 163 & 164 & 165 & 166 & 167 & 168 & 169 & 170 & 171 & 172 & 173 & 174 & 175 & 176 & 177 & 178 & 179 & 180 & 181 & 182 & 183 & 184 & 185 & 186 & 187 & 188 & 189 & 190 & 191 & 192 & 193 & 194 & 195 & 196 & 197 & 198 & 199 & 200 & 201 & 202 & 203 & 204 & 205 & 206 & 207 & 208 & 209 & 210 & 211 & 212 & 213 & 214 & 215 & 216 & 217 & 218 & 219 & 220 & 221 & 222 & 223 & 224 & 225 & 226 & 227 & 228 & 229 & 230 & 231 & 232 & 233 & 234 & 235 & 236 & 237 & 238 & 239 & 240 & 241 & 242 & 243 & 244 & 245 & 246 & 247 & 248 & 249 & 250 & 251 & 252 & 253 & 254 & 255 & 256 & 257 & 258 & 259 & 260 & 261 & 262 & 263 & 264 & 265 & 266 & 267 & 268 & 269 & 270 & 271 & 272 & 273 & 274 & 275 & 276 & 277 & 278 & 279 & 280 & 281 & 282 & 283 & 284 & 285 & 286 & 287 & 288 & 289 & 290 & 291 & 292 & 293 & 294 & 295 & 296 & 297 & 298 & 299 & 300 & 301 & 302 & 303 & 304 & 305 & 306 & 307 & 308 & 309 & 310 & 311 & 312 & 313 & 314 & 315 & 316 & 317 & 318 & 319 & 320 & 321 & 322 & 323 & 324 & 325 & 326 & 327 & 328 & 329 & 330 & 331 & 332 & 333 & 334 & 335 & 336 & 337 & 338 & 339 & 340 & 341 & 342 & 343 & 344 & 345 & 346 & 347 & 348 & 349 & 350 & 351 & 352 & 353 & 354 & 355 & 356 & 357 & 358 & 359 & 360 & 361 & 362 & 363 & 364 & 365 & 366 & 367 & 368 & 369 & 370 & 371 & 372 & 373 & 374 & 375 & 376 & 377 & 378 & 379 & 380 & 381 & 382 & 383 & 384 & 385 & 386 & 387 & 388 & 389 & 390 & 391 & 392 & 393 & 394 & 395 & 396 & 397 & 398 & 399 & 400 & 401 & 402 & 403 & 404 & 405 & 406 & 407 & 408 & 409 & 410 & 411 & 412 & 413 & 414 & 415 & 416 & 417 & 418 & 419 & 420 & 421 & 422 & 423 & 424 & 425 & 426 & 427 & 428 & 429 & 430 & 431 & 432 & 433 & 434 & 435 & 436 & 437 & 438 & 439 & 440 & 441 & 442 & 443 & 444 & 445 & 446 & 447 & 448 & 449 & 450 & 451 & 452 & 453 & 454 & 455 & 456 & 457 & 458 & 459 & 460 & 461 & 462 & 463 & 464 & 465 & 466 & 467 & 468 & 469 & 470 & 471 & 472 & 473 & 474 & 475 & 476 & 477 & 478 & 479 & 480 & 481 & 482 & 483 & 484 & 485 & 486 & 487 & 488 & 489 & 490 & 491 & 492 & 493 & 494 & 495 & 496 & 497 & 498 & 499 & 500 & 501 & 502 & 503 & 504 & 505 & 506 & 507 & 508 & 509 & 510 & 511 & 512 & 513 & 514 & 515 & 516 & 517 & 518 & 519 & 520 & 521 & 522 & 523 & 524 & 525 & 526 & 527 & 528 & 529 & 530 & 531 & 532 & 533 & 534 & 535 & 536 & 537 & 538 & 539 & 540 & 541 & 542 & 543 & 544 & 545 & 546 & 547 & 548 & 549 & 550 & 551 & 552 & 553 & 554 & 555 & 556 & 557 & 558 & 559 & 560 & 561 & 562 & 563 & 564 & 565 & 566 & 567 & 568 & 569 & 570 & 571 & 572 & 573 & 574 & 575 & 576 & 577 & 578 & 579 & 580 & 581 & 582 & 583 & 584 & 585 & 586 & 587 & 588 & 589 & 590 & 591 & 592 & 593 & 594 & 595 & 596 & 597 & 598 & 599 & 600 & 601 & 602 & 603 & 604 & 605 & 606 & 607 & 608 & 609 & 610 & 611 & 612 & 613 & 614 & 615 & 616 & 617 & 618 & 619 & 620 & 621 & 622 & 623 & 624 & 625 & 626 \\ 
  \hline
sort.unique.surgeryplastic16\_nocpts...cpt..... & 11000.00 & 11001.00 & 11004.00 & 11005.00 & 11006.00 & 11008.00 & 11010.00 & 11011.00 & 11012.00 & 11042.00 & 11043.00 & 11044.00 & 11960.00 & 14301.00 & 14302.00 & 15150.00 & 15155.00 & 15200.00 & 15201.00 & 15220.00 & 15221.00 & 15240.00 & 15260.00 & 15261.00 & 15271.00 & 15272.00 & 15273.00 & 15274.00 & 15275.00 & 15277.00 & 15278.00 & 15570.00 & 15572.00 & 15574.00 & 15576.00 & 15600.00 & 15610.00 & 15620.00 & 15630.00 & 15650.00 & 15731.00 & 15732.00 & 15734.00 & 15736.00 & 15738.00 & 15740.00 & 15750.00 & 15756.00 & 15757.00 & 15758.00 & 15760.00 & 15770.00 & 15777.00 & 15830.00 & 15840.00 & 15841.00 & 15842.00 & 15845.00 & 15847.00 & 15920.00 & 15931.00 & 15933.00 & 15934.00 & 15935.00 & 15936.00 & 15937.00 & 15940.00 & 15941.00 & 15944.00 & 15945.00 & 15946.00 & 15950.00 & 15952.00 & 15956.00 & 15958.00 & 15999.00 & 19020.00 & 19110.00 & 19120.00 & 19125.00 & 19260.00 & 19271.00 & 19296.00 & 19301.00 & 19302.00 & 19303.00 & 19304.00 & 19305.00 & 19306.00 & 19307.00 & 19330.00 & 19355.00 & 19396.00 & 19499.00 & 20005.00 & 20102.00 & 20103.00 & 20816.00 & 20900.00 & 20902.00 & 20910.00 & 20920.00 & 20926.00 & 20955.00 & 20962.00 & 21010.00 & 21025.00 & 21026.00 & 21029.00 & 21034.00 & 21040.00 & 21044.00 & 21045.00 & 21047.00 & 21049.00 & 21050.00 & 21070.00 & 21110.00 & 21120.00 & 21121.00 & 21137.00 & 21138.00 & 21141.00 & 21142.00 & 21143.00 & 21145.00 & 21146.00 & 21147.00 & 21150.00 & 21172.00 & 21175.00 & 21181.00 & 21182.00 & 21196.00 & 21198.00 & 21206.00 & 21208.00 & 21209.00 & 21215.00 & 21230.00 & 21235.00 & 21243.00 & 21244.00 & 21248.00 & 21249.00 & 21256.00 & 21260.00 & 21267.00 & 21270.00 & 21275.00 & 21299.00 & 21343.00 & 21344.00 & 21346.00 & 21347.00 & 21355.00 & 21356.00 & 21360.00 & 21365.00 & 21385.00 & 21386.00 & 21390.00 & 21395.00 & 21406.00 & 21407.00 & 21422.00 & 21423.00 & 21433.00 & 21435.00 & 21445.00 & 21454.00 & 21461.00 & 21462.00 & 21465.00 & 21470.00 & 21499.00 & 21501.00 & 21600.00 & 21620.00 & 21750.00 & 21899.00 & 22551.00 & 22612.00 & 22856.00 & 22999.00 & 23140.00 & 23395.00 & 23397.00 & 23405.00 & 23406.00 & 23430.00 & 24000.00 & 24075.00 & 24076.00 & 24077.00 & 24105.00 & 24201.00 & 24305.00 & 24330.00 & 24341.00 & 24342.00 & 24346.00 & 24358.00 & 24359.00 & 24495.00 & 24586.00 & 24666.00 & 24900.00 & 24925.00 & 24930.00 & 25000.00 & 25001.00 & 25020.00 & 25023.00 & 25024.00 & 25025.00 & 25035.00 & 25040.00 & 25073.00 & 25076.00 & 25085.00 & 25101.00 & 25105.00 & 25107.00 & 25110.00 & 25111.00 & 25112.00 & 25115.00 & 25116.00 & 25118.00 & 25119.00 & 25120.00 & 25130.00 & 25135.00 & 25150.00 & 25210.00 & 25215.00 & 25230.00 & 25240.00 & 25248.00 & 25260.00 & 25263.00 & 25265.00 & 25270.00 & 25272.00 & 25274.00 & 25275.00 & 25280.00 & 25295.00 & 25301.00 & 25310.00 & 25312.00 & 25315.00 & 25320.00 & 25332.00 & 25337.00 & 25360.00 & 25390.00 & 25400.00 & 25405.00 & 25431.00 & 25440.00 & 25441.00 & 25442.00 & 25445.00 & 25446.00 & 25447.00 & 25449.00 & 25515.00 & 25545.00 & 25575.00 & 25607.00 & 25608.00 & 25609.00 & 25628.00 & 25645.00 & 25685.00 & 25695.00 & 25900.00 & 25905.00 & 25909.00 & 25920.00 & 25927.00 & 25931.00 & 26117.00 & 26350.00 & 26352.00 & 26356.00 & 26357.00 & 26358.00 & 26370.00 & 26372.00 & 26373.00 & 26390.00 & 26392.00 & 26410.00 & 26412.00 & 26415.00 & 26418.00 & 26420.00 & 26426.00 & 26428.00 & 26433.00 & 26434.00 & 26437.00 & 26440.00 & 26442.00 & 26445.00 & 26449.00 & 26450.00 & 26455.00 & 26460.00 & 26471.00 & 26477.00 & 26478.00 & 26480.00 & 26483.00 & 26485.00 & 26490.00 & 26492.00 & 26497.00 & 26498.00 & 26500.00 & 26502.00 & 26520.00 & 26525.00 & 26530.00 & 26531.00 & 26535.00 & 26536.00 & 26540.00 & 26541.00 & 26542.00 & 26545.00 & 26546.00 & 26548.00 & 26551.00 & 26560.00 & 26561.00 & 26565.00 & 26567.00 & 26593.00 & 26615.00 & 26650.00 & 26665.00 & 26676.00 & 26685.00 & 26686.00 & 26706.00 & 26715.00 & 26727.00 & 26735.00 & 26746.00 & 26765.00 & 26776.00 & 26785.00 & 26910.00 & 26951.00 & 26952.00 & 26989.00 & 26990.00 & 27033.00 & 27047.00 & 27070.00 & 27071.00 & 27080.00 & 27086.00 & 27100.00 & 27122.00 & 27158.00 & 27295.00 & 27299.00 & 27301.00 & 27310.00 & 27327.00 & 27328.00 & 27355.00 & 27365.00 & 27385.00 & 27418.00 & 27447.00 & 27580.00 & 27590.00 & 27596.00 & 27603.00 & 27615.00 & 27618.00 & 27619.00 & 27635.00 & 27640.00 & 27685.00 & 27707.00 & 27880.00 & 27881.00 & 27882.00 & 27884.00 & 27886.00 & 27893.00 & 27899.00 & 28002.00 & 28003.00 & 28805.00 & 29827.00 & 29844.00 & 29845.00 & 29846.00 & 29847.00 & 29888.00 & 29999.00 & 31395.00 & 31760.00 & 31780.00 & 31825.00 & 32820.00 & 35001.00 & 35045.00 & 35180.00 & 35206.00 & 35207.00 & 35226.00 & 35231.00 & 35236.00 & 35301.00 & 35665.00 & 35741.00 & 35820.00 & 35860.00 & 35876.00 & 37765.00 & 37799.00 & 38308.00 & 38542.00 & 38550.00 & 38720.00 & 38724.00 & 38740.00 & 38745.00 & 38760.00 & 38999.00 & 39010.00 & 40500.00 & 40510.00 & 40520.00 & 40525.00 & 40527.00 & 40530.00 & 40650.00 & 40652.00 & 40654.00 & 40700.00 & 40720.00 & 40800.00 & 40804.00 & 40805.00 & 40810.00 & 40812.00 & 40814.00 & 40816.00 & 40830.00 & 40899.00 & 41000.00 & 41007.00 & 41008.00 & 41016.00 & 41017.00 & 41112.00 & 41120.00 & 41130.00 & 41135.00 & 41145.00 & 41155.00 & 41599.00 & 41820.00 & 41826.00 & 41827.00 & 41830.00 & 42120.00 & 42205.00 & 42210.00 & 42215.00 & 42225.00 & 42226.00 & 42410.00 & 42415.00 & 42420.00 & 42425.00 & 42426.00 & 42810.00 & 42815.00 & 42842.00 & 42950.00 & 43100.00 & 43620.00 & 43775.00 & 44005.00 & 44120.00 & 44145.00 & 44187.00 & 44320.00 & 44950.00 & 44970.00 & 46040.00 & 47562.00 & 47563.00 & 49010.00 & 49203.00 & 49204.00 & 49215.00 & 49250.00 & 49322.00 & 49505.00 & 49560.00 & 49561.00 & 49565.00 & 49566.00 & 49568.00 & 49585.00 & 49587.00 & 49606.00 & 49650.00 & 49652.00 & 49656.00 & 49900.00 & 49904.00 & 49906.00 & 49999.00 & 50545.00 & 50920.00 & 52234.00 & 53430.00 & 53520.00 & 54322.00 & 54440.00 & 54660.00 & 55100.00 & 55175.00 & 55866.00 & 55970.00 & 55980.00 & 56405.00 & 56420.00 & 56620.00 & 56625.00 & 56630.00 & 56634.00 & 56800.00 & 56805.00 & 56810.00 & 57110.00 & 57240.00 & 57250.00 & 57260.00 & 57267.00 & 57288.00 & 57291.00 & 57292.00 & 57335.00 & 57425.00 & 58145.00 & 58150.00 & 58180.00 & 58210.00 & 58260.00 & 58262.00 & 58263.00 & 58270.00 & 58291.00 & 58550.00 & 58552.00 & 58553.00 & 58554.00 & 58571.00 & 58572.00 & 58573.00 & 58700.00 & 58720.00 & 58740.00 & 58925.00 & 58940.00 & 58953.00 & 58957.00 & 59120.00 & 59151.00 & 60240.00 & 60500.00 & 61304.00 & 61500.00 & 61501.00 & 61552.00 & 61563.00 & 61586.00 & 61605.00 & 61619.00 & 61885.00 & 61888.00 & 63276.00 & 64702.00 & 64704.00 & 64708.00 & 64712.00 & 64713.00 & 64722.00 & 64726.00 & 64727.00 & 64821.00 & 64831.00 & 64832.00 & 64834.00 & 64835.00 & 64836.00 & 64856.00 & 64857.00 & 64861.00 & 64864.00 & 64865.00 & 64866.00 & 64868.00 & 64886.00 & 64890.00 & 64891.00 & 64892.00 & 64893.00 & 64895.00 & 64898.00 & 64901.00 & 64905.00 \\ 
   \hline
\end{tabular}
\caption{CPTs in Surgery Specialty Plastic (without given CPTs)} 
\end{table}
